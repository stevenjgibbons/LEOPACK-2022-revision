\clearpage
\section{Information on key subroutines}
\label{app:appkeysubs}

\subsection{SUBROUTINE FDCMBD}
\label{sec:fdcmbdinfo}

{\bf F}inite {\bf D}ifference
{\bf C}oefficient {\bf M}atrix {\bf B}uil{\bf D}. \newline

{\bf Calling sequence}:-
\begin{verbatim}
FDCMBD( NR, NBN, NLMN, NRMN, NLMC, NRMC, NCFM,
        NFDCM, NDRVS, NDRVM, IWORK, XARR, FDCM,
        COEFM, WORK1, WORK2 )
\end{verbatim}

{\bf Purpose}:- \newline

If \verb+XARR+ is a double precision array of length
\verb+NR+ such that \verb+XARR( j )+$=x_j$, then
{\bf FDCMBD } builds a double precision array, \verb+FDCM+,
of dimension \newline
\verb+( NFDCM, NR, NDRVM )+ such that
for a given node number, $j$, the $n^{\rm th}$
derivative of $f(x)$ is given by
\bed
f^{(n)}( x_j ) = \sum_{i= {\rm LN}}^{\rm RM}
{\rm FDCM}( k, j, n ) f( x_i )
\eed
where
\begin{verbatim}
LN = MAX( NLMC, j - NBN )
RN = MIN( NRMC, j + NBN ),
\end{verbatim}
\verb+NBN+ is the number of nodes on each side
for central difference formulae,
\begin{verbatim}
k = i - j + NBN + 1,
\end{verbatim}
and \verb+NLMC+ and \verb+NRMC+ are respectively the
left most and right most nodes which may be used to
obtain a finite difference formula.
The {\rm FDCM}$( k, j, n )$ are calculated for all
$j$ with \verb+NLMN+$ \leq j \leq$
\verb+NRMN+. Elements corresponding to other
$j$ are not referred to.

The leading dimension of \verb+FDCM+ must be atleast
\verb+(2*NBN+$+$\verb+1)+ as must the dimension
\verb+NCFM+.

The maximum value of $n$ calculated is specified by
the parameter \verb+NDRVS+ which must be equal to or
less than the value \verb+NDRVM+. The value of
\verb+NDRVS+ is limited by the bandwidth as specified
by \verb+NBN+. Let
\begin{verbatim}
NLCS = NLMN - NLMC
NRCS = NRMC - NRMN
\end{verbatim}
and let
\begin{verbatim}
I = MIN( NLCS, NRCS) + NBN.
\end{verbatim}
\verb+NDRVS+ must then be no greater than \verb+I+.

The maximum value of $n$ calculated is specified by
the parameter \verb+NDRVS+ which must be equal to or
less than the value \verb+NDRVM+. The value of
\verb+NDRVS+ is limited by the bandwidth as specified
by \verb+NBN+. Let
\begin{verbatim}
NLCS = NLMN - NLMC
NRCS = NRMC - NRMN
\end{verbatim}
and let
\begin{verbatim}
I = MIN( NLCS, NRCS) + NBN.
\end{verbatim}
\verb+NDRVS+ must then be no greater than \verb+I+.

{\bf Arguments}:- \newline

\begin{tabular}{|l|l|l|}
\hline
\verb+ NR  + & Int. scal. & Number of radial grid nodes. \\
\hline
\verb+ NBN   + & Int. scal. & Number of bounding nodes. \\
\hline
\verb+ NLMN + & Int. scal. & Left-most (row) node. See above.\\
\hline
\verb+ NRMN + & Int. scal. & Right-most (row) node. See above.\\
\hline
\verb+ NLMC + & Int. scal. & Left-most (column) node. See above.\\
\hline
\verb+ NRMC + & Int. scal. & Right-most (column) node. See above.\\
\hline
\verb+ NCFM + & Int. scal. & Leading dimension of
\verb+COEFM+. \\
& & Must be atleast (2*NBN+1). \\
\hline
\verb+ NFDCM + & Int. scal. & Leading dimension of
\verb+FDCM+. \\
& & Must be atleast (2*NBN+1). \\
\hline
\verb+ NDVRS + & Int. scal. & Highest derivative to be calculated. \\
& & Must not exceed \verb+NDVRM+ and see above. \\
\hline
\verb+ NDVRM + & Int. scal. & Maximum number of derivatives \\
& & which can be stored in memory. \\
\hline
\verb+ IWORK + & Int. arr. & Dimension ( NCFM ). Work array. \\
\hline
\verb+ XARR + & D.p. arr. & Values of $x$. Dimension( NR ). \\
\hline
\verb+ FDCM + & D.p. arr. & Finite difference coefficients. \\
& & Dimension( NFDCM, NR, NDRVM ). See above. \\
\hline
\verb+ COEFM + & D.p. arr. & Dimension (NCFM, NCFM ). Work array. \\
\hline
\verb+ WORK1 + & D.p. arr. & Dimension (NCFM). Work array. \\
\hline
\verb+ WORK2 + & D.p. arr. & Dimension (NCFM). Work array. \\
\hline
\end{tabular} \newline

{\bf Subroutines and functions called}:- \newline

\verb+GFDCFD+.

\clearpage
\subsection{SUBROUTINE GFDCFD}
\label{sec:gfdcfdinfo}

{\bf G}eneral {\bf F}inite {\bf D}ifference
{\bf C}oefficient {\bf F}in{\bf D}. \newline

{\bf Calling sequence}:-
\begin{verbatim}
GFDCFD ( X0, XARR, NNDS, COEFM, NCFM, IPCM, WORK)
\end{verbatim}

{\bf Purpose}:- \newline

If the double precision array \verb+XARR+ contains
\verb+NNDS+ distinct values of $x$  \newline
(\verb+XARR( j ) =+ $x_j$) then, for a
given value $x_0$, {\bf GFDCFD } returns the coefficients
$c^n_{j, x_0}$ such that if $f$ is a function of $x$,
then
\bed
f^n( x_0 ) \approx  \sum_{j = 1}^{\rm NNDS}
\left(
c^n_{j, x_0} f( x_j )
\right).
\eed
Here, $f^n( x_0 )$ denotes the $n^{\rm th}$ derivative
of $f$ with respect to $x$, evaluated at $x_0$.
The coefficient $c^n_{j, x_0}$ is returned in
the array element
\verb+COEFM( n +$+$\verb+ 1, J )+.
The $n=0$ coefficients are returned in the first
row of the matrix and, although these are not
very useful if $x_0$ corresponds to one of the
$x_j$, they allow a useful way of interpolating
a function $f$ to $x=x_0$, if $x_0$ is distinct
from all the $x_j$.

The higest derivative obtainable from
\verb+NNDS+ distinct values of $x$ is \verb+NNDS-1+
although this estimate is likely to be highly
inaccurate. For arbitrarily spaced $x$, there is
no easy way of quantifying the size of the error.

The routine checks to ensure that the
\verb+XARR( j )+ are all distinct, otherwise
it would be attempted to solve a singular matrix.


{\bf Arguments}:- \newline

\begin{tabular}{|l|l|l|}
\hline
\verb+ X0  + & D.p. scal. & Value of $x_0$ where deriv.s are
                            required. \\
\hline
\verb+ XARR  + & D.p. arr. & Dimension (NNDS). See above. \\
\hline
\verb+ NNDS + & Int. scal. & Number of grid nodes. \\
\hline
\verb+ COEFM + & D.p. arr. & Dimension (NCFM, NCFM ). See above. \\
\hline
\verb+ NCFM  + & Int. scal. & Leading dimension of matrix COEFM. \\
& & Must be greater than or equal to \verb+NNDS+. \\
\hline
\verb+ IPCM + & Int. arr. & Dimension (NCFM). Work array. \\
\hline
\verb+ WORK + & D.p. arr. & Dimension (NCFM). Work array. \\
\hline
\end{tabular} \newline

{\bf Subroutines and functions called}:- \newline

\verb+MATOP+ and the {\bf LAPACK} routines
\verb+DGETRF+ and \verb+DGETRI+.

\subsection{SUBROUTINE HMFRD}
\label{sec:hmfrdinfo}

{\bf H}ar{\bf M}onic
{\bf F}ile
{\bf R}ea{\bf D}. \newline

{\bf Calling sequence}:-
\begin{verbatim}
HMFRD( NH, NHMAX, MHT, MHL, MHM, MHP, NCUDS, NDCS,
       MHIBC, MHOBC, LARR, LU, FNAME )
\end{verbatim}

{\bf Purpose}:- \newline

Reads the indices of the spherical harmonic set (along
with the boundary conditions) from a file specified by the
filename \verb+FNAME+.

The first line of the new file will contain only \verb+NH+,
the number of spherical harmonics.
The next \verb+NH+ lines will all consist of the 5 integer
values
\begin{verbatim}
MT   ML   MM    IIBF  IOBF
\end{verbatim}

\verb+MT+ contains one of the following values which describes
the function :-
\begin{verbatim}
MT = 1 for a poloidal velocity harmonic.
MT = 2 for a toroidal velocity harmonic.
MT = 3 for a temperature harmonic.
MT = 4 for a poloidal magnetic field harmonic.
MT = 5 for a toroidal magnetic field harmonic.
\end{verbatim}

\verb+ML+ contains the spherical harmonic degree, $l$, and
\verb+MM+ will contain $m$,
the spherical harmonic order, for $\cos m \phi$ dependence
and $-m$ for $\sin m \phi$ dependence.

\verb+IIBF+ and \verb+IOBF+ describe the boundary condition
required for that radial function at the inner and outer 
boundaries respectively. (See {\bf SVFDCF},
section (\ref{sec:svfdcfinfo}), for details.)
They are the values of \verb+MHIBC+ and \verb+MHOBC+ which
correspond to the value of \verb+MHP(ih)+.

The arrays \verb+LARR+, \verb+MHIBC+ and \verb+MHOBC+
required by {\bf SVFDCF} are calculated in situ in the
program. On input, \verb+NCUDS+ is the number of currently
used differencing schemes (possibly none) and on output,
this number will be modified to include those schemes
demanded by the new set of spherical harmonics.
All difference schemes which are not yet needed are
labelled with \verb+LARR( is ) = -1 + as required by
{\bf SVFDCF}.

\clearpage
{\bf Arguments}:- \newline

\begin{tabular}{|l|l|l|}
\hline
\verb+ NH + & Int. scal. & Number of spherical harmonics. \\
\hline
\verb+ NHMAX + & Int. scal. & Maximum number of spherical harmonics. \\
\hline
\verb+ MHT + & Int. arr. & Dim (NHMAX). \\
& & \verb+MHT(ih) = MT+ for harmonic number \verb+ih+. \\
\hline
\verb+ MHL + & Int. arr. & Dim (NHMAX). \\
& & \verb+MHL(ih) = ML+ for harmonic number \verb+ih+. \\
\hline
\verb+ MHM + & Int. arr. & Dim (NHMAX). \\
& & \verb+MHM(ih) = MM+ for harmonic number \verb+ih+. \\
\hline
\verb+ MHP + & Int. arr. & Dim (NHMAX). \\
& & \verb+MHP( ih )+ contains the index of the arrays \\
& & \verb+MHIBC+ and \verb+MHOBC+ which correspond to \\
& & harmonic \verb+ih+. \\
\hline
\verb+ NCUDS + & Int. scal. & Number of currently used finite \\
& & difference schemes. \\
\hline
\verb+ NDCS + & Int. scal. & Maximum distinct finite \\
& & difference schemes. See {\bf SVFDCF}. \\
\hline
\verb+ MHIBC + & Int. arr. & Dimension \verb+( NDCS )+. \verb+MHIBC( is )+
contains \\
& & a flag indicating how the inner boundary must \\
& & be treated in scheme \verb+is+. See {\bf SVFDCF}. \\
\hline
\verb+ MHOBC + & Int. arr. & Dimension \verb+( NDCS )+. \verb+MHOBC( is )+
contains \\
& & a flag indicating how the outer boundary must \\
& & be treated in scheme \verb+is+. See {\bf SVFDCF}. \\
\hline
\verb+ LARR + & Int. arr. & Dimension \verb+( NDCS )+. \\
& & Auxilliary array for {\bf SVFDCF}. \\
\hline
\verb+ LU + & Int. scal. & Logical file unit number. \\
\hline
\verb+ FNAME + & Char(*) & Filename. \\
\hline
\end{tabular} \newline

{\bf Subroutines and functions called}:- \newline

\verb+FOPEN+ and
\verb+FCLOSE+.

\clearpage
\subsection{SUBROUTINE HMFWT}
\label{sec:hmfwtinfo}

{\bf H}ar{\bf M}onic
{\bf F}ile
{\bf W}ri{\bf T}e. \newline

{\bf Calling sequence}:-
\begin{verbatim}
HMFWT( NH, MHT, MHL, MHM, MHP, NDCS, MHIBC, MHOBC, LU, FNAME )
\end{verbatim}

{\bf Purpose}:- \newline

Wries the indices of the spherical harmonic set (along
with the boundary conditions) to a file specified by the
filename \verb+FNAME+.

The first line of the new file will contain only \verb+NH+,
the number of spherical harmonics.
The next \verb+NH+ lines will all consist of the 5 integer
values
\begin{verbatim}
MT   ML   MM    IIBF  IOBF
\end{verbatim}

\verb+MT+ contains one of the following values which describes
the function :-
\begin{verbatim}
MT = 1 for a poloidal velocity harmonic.
MT = 2 for a toroidal velocity harmonic.
MT = 3 for a temperature harmonic.
MT = 4 for a poloidal magnetic field harmonic.
MT = 5 for a toroidal magnetic field harmonic.
\end{verbatim}

\verb+ML+ contains the spherical harmonic degree, $l$, and
\verb+MM+ will contain $m$,
the spherical harmonic order, for $\cos m \phi$ dependence
and $-m$ for $\sin m \phi$ dependence.

\verb+IIBF+ and \verb+IOBF+ describe the boundary condition
required for that radial function at the inner and outer 
boundaries respectively. (See {\bf SVFDCF},
section (\ref{sec:svfdcfinfo}), for details.)
They are the values of \verb+MHIBC+ and \verb+MHOBC+ which
correspond to the value of \verb+MHP(ih)+.

{\bf Arguments}:- \newline

\begin{tabular}{|l|l|l|}
\hline
\verb+ NH + & Int. scal. & Number of spherical harmonics. \\
\hline
\verb+ MHT + & Int. arr. & Dim (*) with length atleast NH. \\
& & \verb+MHT(ih) = MT+ for harmonic number \verb+ih+. \\
\hline
\verb+ MHL + & Int. arr. & Dim (*) with length atleast NH. \\
& & \verb+MHL(ih) = MT+ for harmonic number \verb+ih+. \\
\hline
\verb+ MHM + & Int. arr. & Dim (*) with length atleast NH. \\
& & \verb+MHM(ih) = MT+ for harmonic number \verb+ih+. \\
\hline
\verb+ MHP + & Int. arr. & Dim (*) with length atleast NH. \\
& & \verb+MHP( ih )+ contains the index of the arrays \\
& & \verb+MHIBC+ and \verb+MHOBC+ which correspond to \\
& & harmonic \verb+ih+. \\
\hline
\verb+ NDCS + & Int. scal. & Maximum distinct finite \\
& & difference schemes. See {\bf SVFDCF}. \\
\hline
\verb+ MHIBC + & Int. arr. & Dimension \verb+( NDCS )+. \verb+MHIBC( is )+
contains \\
& & a flag indicating how the inner boundary must \\
& & be treated in scheme \verb+is+. See {\bf SVFDCF}. \\
\hline
\verb+ MHOBC + & Int. arr. & Dimension \verb+( NDCS )+. \verb+MHOBC( is )+
contains \\
& & a flag indicating how the outer boundary must \\
& & be treated in scheme \verb+is+. See {\bf SVFDCF}. \\
\hline
\verb+ LU + & Int. scal. & Logical file unit number. \\
\hline
\verb+ FNAME + & Char(*) & Filename. \\
\hline
\end{tabular} \newline

{\bf Subroutines and functions called}:- \newline

\verb+FOPEN+ and
\verb+FCLOSE+.

\subsection{SUBROUTINE LDGNMF}
\label{sec:ldgnmfinfo}

{\bf L}inear
{\bf D}ependence of
{\bf G}rid
{\bf N}ode
{\bf M}atrix
{\bf F}orm. \newline

{\bf Calling sequence}:-
\begin{verbatim}
LDGNMF( NR, NNDS, NALF, NARF, L, IIBC, IOBC, NCFM, 
        XARR, AMAT, WMAT, WORK1, WORK2, IPCM )
\end{verbatim}

{\bf Purpose}:- \newline

Let $f_j$ denote the function $f$ evaluated
at $x_j$, the value stored in \verb+XARR( j )+. \newline
Let $f_j^{(m)}$ denote the $m^{\rm th}$ derivative
with respect to $x$ evaluated at $x_j$. \newline
Then a finite difference scheme may be formulated
such that
\bed
\left(
\begin{array}{c}
f_j^{(0)} \\
f_j^{(1)} \\
\vdots \\
f_j^{(n-1)} \\
\end{array}
\right)
= 
\left(
\begin{array}{cccc}
 c^{(0)}_{k+1}  &  c^{(0)}_{k+2}  &  \cdots  &  c^{(0)}_{k+n}  \\
 c^{(1)}_{k+1}  &  c^{(1)}_{k+2}  &  \cdots  &  c^{(1)}_{k+n}  \\
 \vdots  & \vdots   &  \cdots  &    \vdots \\
 c^{(n-1)}_{k+1}  &  c^{(n-1)}_{k+2}  &  \cdots  &  c^{(n-1)}_{k+n}  \\
\end{array}
\right)
\left(
\begin{array}{c}
f_{k+1} \\
f_{k+2} \\
\vdots \\
f_{k+n} \\
\end{array}
\right).
\eed
The matrix ${\bm C}$ above, containing the elements
$c^{(m)}_{k+i}$, may be formed by a call to {\bf GFDCFD}
with the calling sequence
\begin{verbatim}
CALL GFDCFD ( XJ, XA, NNDS, COEFM, NCFM, IPCM, WORK)
\end{verbatim}
where \verb+XJ+ gives the value of $x_j$,
\verb+XA( i )+  gives the value of $x_{k+i}$,
\verb+NNDS+ is $n$ and
\verb+COEFM( +$m + 1$\verb+, i )+ gives the
value of $c^{(m)}_{k+i}$.

However, if boundary conditions must be satisfied
at $x_1$ and $x_{\rm NR}$ (\verb+XARR( 1 )+
and \verb+XARR( NR )+ respectively) then the
values $f_{k+1}$, $f_{k+2}$, $\cdots$, $f_{k+n}$
may not be linearly independent. In other words,
the equation above becomes
\bed
\left(
\begin{array}{c}
f_j^{(0)} \\
f_j^{(1)} \\
\vdots \\
f_j^{(n-1)} \\
\end{array}
\right)
= 
\left(
{\bm C}
\right)
\left(
{\bm A}
\right)
\left(
\begin{array}{c}
f_{k+1} \\
f_{k+2} \\
\vdots \\
f_{k+n} \\
\end{array}
\right),
\eed
where the matrix ${\bm A}$ defines the dependency between
the different $f_{k+i}$. If the boundary condition does
not effect the nodes $k+1$, $k+2$ to $k+n$ (to the
accuracy of the finite differencing scheme) then
${\bm A}$ is clearly the identity matrix. 
Otherwise, ${\bm A}$ is a matrix of Rank $ < n $,
and the purpose of {\bf LDGNMF} is to calculate this
matrix.
The element $f_{k+i}$ is expressed in terms of the
$f_s$ (with $s \in \{ k+1, k+2, \cdots , k+n \}$) by
\bed
f_{k+i} = \sum_{s=1}^n d_{i,s} f_{k+s}.
\eed

The integer values \verb+NALF+ and \verb+NARF+ are
respectively the number of nodes at the left and
the right for which the value is a linear combination of
the values at the other nodes. 
Both \verb+NALF+ and \verb+NARF+ {\bf MUST} take one
of the values \verb+0+, \verb+1+ or \verb+2+.
(This is because there is no case in the standard
MHD equations where we need to apply more than 2
boundary conditions at a particular boundary.)
If both \verb+NALF+ and \verb+NARF+ are zero then
${\bm A}$ is returned as ${\bm I}_n$.
If either \verb+NALF+ or \verb+NARF+ is non-zero
then the other must be zero! (Otherwise the
finite difference scheme is not going to work). \newline
If \verb+NALF = 1+ then $f_{k+1}$ is a
linear sum of $\{ f_{k+2}, \cdots , f_{k+n} \}$. \newline
If \verb+NALF = 2+ then $f_{k+1}$ and $f_{k+2}$ are both 
linear sums of $\{ f_{k+3}, \cdots , f_{k+n} \}$. \newline
If \verb+NARF = 1+ then $f_{k+n}$ is a
linear sum of $\{ f_{k+1}, \cdots , f_{k+n-1} \}$. \newline
If \verb+NARF = 2+ then $f_{k+n}$ and $f_{k+n-1}$ are both 
linear sums of $\{ f_{k+1}, \cdots , f_{k+n-2} \}$. \newline

Now the actual form of the boundary condition
can be one of many different cases. This is 
determined by the flags \verb+IIBC+ and 
\verb+IOBC+, which specify the inner and outer
boundary condition respectively.

The following table gives the possibilities for
\verb+IIBC+ :- \newline

\begin{tabular}{|c|l|}
\hline
\verb+IIBC+ & Inner boundary condition. \\
\hline
\hline
\verb+1+ & None imposed \\
\hline
\verb+2+ & Function must vanish at boundary. \\
\hline
\verb+3+ & First derivative must vanish at boundary. \\
\hline
\verb+4+ & Both function and first derivative must vanish. \\
\hline
\verb+5+ & Both function and second derivative must vanish. \\
\hline
\verb+6+ & $rdf/dr - f(r) = 0$ \\
\hline
\verb+7+ & $r df/dr - l f(r) = 0$ with $l=$\verb+L+. \\
\hline
\end{tabular} \newline

The following table gives the possibilities for
\verb+IOBC+ :- \newline

\begin{tabular}{|c|l|}
\hline
\verb+IOBC+ & Outer boundary condition. \\
\hline
\hline
\verb+1+ & None imposed \\
\hline
\verb+2+ & Function must vanish at boundary. \\
\hline
\verb+3+ & First derivative must vanish at boundary. \\
\hline
\verb+4+ & Both function and first derivative must vanish. \\
\hline
\verb+5+ & Both function and second derivative must vanish. \\
\hline
\verb+6+ & $rdf/dr - f(r) = 0$ \\
\hline
\verb+7+ & $r df/dr + (l+1) f(r) = 0$ with $l=$\verb+L+. \\
\hline
\end{tabular} \newline

{\bf Arguments}:- \newline

\begin{tabular}{|l|l|l|}
\hline
\verb+ NR + & Int. scal. & Total number of radial grid nodes. \\
\hline
\verb+ NNDS + & Int. scal. & Number of nodes which are used \\
& & to take derivative ($n$ above). \\
\hline
\verb+ NALF + & Int. scal. & Number of nodes at the left \\
& & which are a linear combination of the others. \\
\hline
\verb+ NARF + & Int. scal. & Number of nodes at the right\\
& & which are a linear combination of the others. \\
\hline
\verb+ L + & Int. scal. & Spherical harmonic degree, $l$. \\
& & Only relevant for magnetic field b.c.s \\
\hline
\verb+ IIBC + & Int. scal. & Inner boundary condition flag. See above. \\
\hline
\verb+ IOBC + & Int. scal. & Outer boundary condition flag. See above. \\
\hline
\verb+ NCFM + & Int. scal. & Leading dimension of matrix \verb+AMAT+. \\
\hline
\verb+ XARR  + & D.p. arr. & Dimension \verb+(NR)+. See above. \\
\hline
\verb+ AMAT + & D.p. arr. & Dimension \verb+(NCFM,NCFM)+. Returns matrix \\
& &  ${\bm A}$ above. \\
\hline
\verb+ WMAT + & D.p. arr. & Dimension \verb+(NCFM,NCFM)+. Work array. \\
\hline
\verb+ WORK1 + & D.p. arr. & Dimension \verb+(NCFM)+. Work array
for {\bf GFDCFD}. \\
\hline
\verb+ WORK2 + & D.p. arr. & Dimension \verb+(NCFM)+. Work array
for {\bf GFDCFD}. \\
\hline
\verb+ IPCM + & Int. arr. & Dimension \verb+(NCFM)+. Work array
for {\bf GFDCFD}. \\
\hline
\end{tabular} \newline

{\bf Subroutines and functions called}:- \newline

\verb+MATOP+, \verb+GFDCFD+.

\subsection{SUBROUTINE SCHNLA}
\label{sec:schnlainfo}

{\bf Calling sequence}:-
\begin{verbatim}
SCHNLA ( PA, DPA, GAUX, LH, NTHPTS )
\end{verbatim}

{\bf Purpose}:- \newline

To evaluate an array of all the associated Legendre Polynomials,
$P_l^m$, up to and including degree $l$=\verb+LH+ at \verb+NTHPTS+
different values of $\theta$.
The derivatives with respect to $\theta$ are also calculated.

\verb+GAUX+ is a double precision array of length
\verb+NTHPTS+ with \verb+GAUX( i )+ containing the
element $\cos \theta_i$.

\verb+PA+ is an array with dimensions
{\bf ( ( LH + 1 )*( LH + 2 )/2 , NTHPTS)} and $P_l^m( \cos \theta_i )$
is stored in the element \verb+PA( K, I )+ where the value of
\verb+K+ is given by the function \verb+KLM( L, M )+:
i.e. \verb.KLM = L*(L+1)/2+M+1..

\verb+DPA+ is arranged identically except that its elements
contain $d P_l^m( \cos \theta_i )/ d\theta$.

$P_l^m$ satisfies the normalisation
\bed
\int_{-1}^{1} \left[ P_l^m(x) \right]^2 dx = \fr{ 2 ( 2 -
\delta_{m0} ) }{ 2 l + 1 }.
\eed

{\bf Arguments}:- \newline

\begin{tabular}{|l|l|l|}
\hline
\verb+ PA + & D.p. arr. & See above. \\
& Dim. {\bf ( N1 , NTHPTS)} & \\
& with {\bf N1 = ( LH + 1 )*( LH + 2 )/2 } & \\
\hline
\verb+ DPA + & D.p. arr. & See above. \\
& Dim. {\bf ( N1 , NTHPTS)} & \\
& with {\bf N1 = ( LH + 1 )*( LH + 2 )/2 } & \\
\hline
\verb+ GAUX + & D.p. arr. & See above \\
& Dim. {\bf ( NTHPTS )} & \\
\hline
\verb+ LH + & Int. scal. & Maximum degree, $l$ \\
\hline
\verb+ NTHPTS + & Int. scal. & Number of $\theta$ points \\
\hline
\end{tabular} \newline

{\bf Subroutines and functions called}:- \newline

\verb+PMM+, \verb+PMM1+ and \verb+PLM+;
\verb+DPMM+, \verb+DPMM1+ and \verb+DPLM+;
\verb+KLM+.

\subsection{SUBROUTINE SVFDCF}
\label{sec:svfdcfinfo}

{\bf S}olution
{\bf V}ector
{\bf F}inite
{\bf D}ifference
{\bf C}oefficient
{\bf F}ind. \newline

{\bf Calling sequence}:-
\begin{verbatim}
SVFDCF( NR, NDCS, NBN, NLMR, NRMR, MHIBC, MHOBC,
        LARR, NCFM, NFDCM, NDRVS, NDRVM, XARR,
        IWORK, SVFDC, COEFM1, COEFM2, WORK1, WORK2 )
\end{verbatim}

{\bf Purpose}:- \newline

Let $f$ be a function of $x$ with \verb+NR+ discrete
values $x_j$ being stored in \verb+XARR( j )+.
{\bf SVFDCF} builds a double precision matrix
\verb+SVFDC+ of dimension \newline
\verb+(NFDCM, NR, NDRVM+$+$\verb+1, NDCS )+
such that for a given node number, $j$, the $n^{\rm th}$
derivative of $f(x)$ is given by
\bed
f^{(n)}( x_j ) = \sum_{i= {\rm LN}}^{\rm RM}
{\rm SVFDC}( k, j, n+1, S ) f( x_i )
\eed
where
\begin{verbatim}
LN = MAX(  1, j - NBN )
RN = MIN( NR, j + NBN ),
\end{verbatim}
\verb+NBN+ is the number of nodes on each side
for central difference formulae,
\begin{verbatim}
k = i - j + NBN + 1,
\end{verbatim}
and $S$ is an integer value describing one of several
different finite-differencing schemes which may
all be stored in \verb+SVFDC+.
The properties of the scheme $S$ (\verb+is+) is described by two
integer parameters, \verb+MHIBC( is )+ and \verb+MHOBC( is )+.
Their values are given in the following two tables.

The possibilities for \verb+MHIBC( is )+ are :- \newline

\begin{tabular}{|c|l|}
\hline
\verb+MHIBC( is )+ & Inner boundary condition. \\
\hline
\hline
\verb+1+ & None imposed \\
\hline
\verb+2+ & Function must vanish at boundary. \\
\hline
\verb+3+ & First derivative must vanish at boundary. \\
\hline
\verb+4+ & Both function and first derivative must vanish. \\
\hline
\verb+5+ & Both function and second derivative must vanish. \\
\hline
\verb+6+ & $rdf/dr - f(r) = 0$ \\
\hline
\verb+7+ & $r df/dr - l f(r) = 0$ with $l=$\verb+L+. \\
\hline
\end{tabular} \newline

The possibilities for \verb+MHOBC( is )+ are :- \newline

\begin{tabular}{|c|l|}
\hline
\verb+MHOBC( is )+ & Outer boundary condition. \\
\hline
\hline
\verb+1+ & None imposed \\
\hline
\verb+2+ & Function must vanish at boundary. \\
\hline
\verb+3+ & First derivative must vanish at boundary. \\
\hline
\verb+4+ & Both function and first derivative must vanish. \\
\hline
\verb+5+ & Both function and second derivative must vanish. \\
\hline
\verb+6+ & $rdf/dr - f(r) = 0$ \\
\hline
\verb+7+ & $r df/dr + (l+1) f(r) = 0$ with $l=$\verb+L+. \\
\hline
\end{tabular} \newline

In schemes with \verb+MHIBC( is ) = 1+, all nodes near
the inner boundary are required for taking derivatives.
In schemes with \verb+MHIBC( is ) = +2, 3, 6 and 7,
the first grid node is essentially omitted from the
solution vector and only nodes $2, 3, \cdots$ are
used for taking derivatives.
In schemes with \verb+MHIBC( is ) = +4 and 5,
the first two grid nodes are omitted and only nodes $3, 4, \cdots$ are
used for taking derivatives.
A similar situation applies at the outer boundary.

Set \verb+LARR( is )+ to the spherical harmonic degree, $l$,
for poloidal magnetic field (insulating boundaries) finite
difference schemes and to \verb+-1+ to make {\bf SVFDCF}
overlook scheme \verb+is+ altogether (maybe to be filled
using a different set of parameters).

{\bf Arguments}:- \newline

\begin{tabular}{|l|l|l|}
\hline
\verb+ NR + & Int. scal. & Total number of radial grid nodes. \\
\hline
\verb+ NDCS + & Int. scal. & Number of distinct differencing \\
& & schemes to be stored in the array \verb+SVFDC+. \\
\hline
\verb+ NBN + & Int. scal. & Maximum number of nodes to left or \\
& & right which may be used to calculate derivatives. \\
\hline
\verb+ NLMR + & Int. scal. & The lowest $j$ for which terms are \\
& & to be calculated for \verb+SVFDC(i,j,ND+$+$\verb+1,K)+. \\
\hline
\verb+ NRMR + & Int. scal. & The highest $j$ for which terms are \\
& & to be calculated for \verb+SVFDC(i,j,ND+$+$\verb+1,K)+. \\
\hline
\verb+ MHIBC + & Int. arr. & Dimension \verb+( NDCS )+. \verb+MHIBC( is )+
contains \\
& & a flag indicating how the inner boundary must \\
& & be treated in scheme \verb+is+. See above list. \\
\hline
\verb+ MHOBC + & Int. arr. & Dimension \verb+( NDCS )+. \verb+MHOBC( is )+
contains \\
& & a flag indicating how the outer boundary must \\
& & be treated in scheme \verb+is+. See above list. \\
\hline
\verb+ LARR + & Int. arr. & Dimension \verb+( NDCS )+. \verb+LARR( is )+
contains \\
& & the spherical harmonic degree, $l$, for scheme \verb+is+. \\
& & If \verb+LARR( is ) = -1+, scheme \verb+is+ is ignored. \\
\hline
\verb+ NCFM + & Int. scal. & Leading dimension of work arrays. \\
& & Must be atleast $2*NBN + 1$. \\
\hline
\verb+ NFDCM + & Int. scal. & Leading dimension of array \verb+SVFDC+. \\
& & Must be atleast $2*NBN + 1$. \\
\hline
\verb+ NDRVS + & Int. scal. & Number of derivatives to be calculated. \\
\hline
\verb+ NDRVM + & Int. scal. & Maximum no. of deriv. coefficients \\
& & which may be stored in \verb+SVFDC+. \\
\hline
\verb+ XARR + & D.p. arr. & Dimension \verb+(NR)+. Radial grid values. \\
\hline
\verb+ IWORK + & Int. arr. & Dimension \verb+(NCFM)+. Work array. \\
\hline
\verb+ SVFDC + & D.p. arr. & Dimension
\verb+(NFDCM, NR, NDRVM+$+$\verb+1, NDCS )+. \\
& & Coefficients. See above \\
\hline
\verb+ COEFM1 + & D.p. arr. & Dimension \verb+( NCFM, NCFM )+. Work array. \\
\hline
\verb+ COEFM2 + & D.p. arr. & Dimension \verb+( NCFM, NCFM )+. Work array. \\
\hline
\verb+ WORK1 + & D.p. arr. & Dimension \verb+( NCFM )+. Work array. \\
\hline
\verb+ WORK2 + & D.p. arr. & Dimension \verb+( NCFM )+. Work array. \\
\hline
\end{tabular} \newline

{\bf Subroutines and functions called}:- \newline

\verb+LDGNMF+, \verb+GFDCFD+, \verb+EMMULT+.
