\clearpage
\section{ Vector spherical harmonics }
\label{app:sphervec}

This appendix is essentially taken straight out of my thesis,
\cite{gibbonsthesis}. However, I think it is a helpful reference
in terms of understanding how the codes work, given that all of
these relations are applied throughout, and therefore worth the
additional space.

\section{ The ${\mathcal {D} }_l$ operator }
\label{sec:dloperator}

The spherical harmonic $Y_{\alpha}$ is defined by
\beq
Y_{\alpha}( \theta, \phi ) = P_{l_{\alpha}}^{m_{\alpha}}( \cos \theta )
\CS_{\alpha} m_{\alpha} \phi,
\label{eq:ylmdef}
\eeq
where the Associated Legendre Function,
$P_{l_{\alpha}}^{m_{\alpha}}( \cos \theta )$, satisfies
\beq
\fr{d}{d \theta} \left( \sin \theta
 \fr{d P_{l_{\alpha}}^{m_{\alpha}}( \cos \theta ) }{ d \theta }
\right) + \left[
l_{\alpha}(l_{\alpha}+1) - \fr{ (m_{\alpha})^2 }{ \sin^2 \theta }
\right]
P_{l_{\alpha}}^{m_{\alpha}}( \cos \theta ) = 0
\label{eq:alpeq}
\eeq
and $Y_{\alpha}$ is Schmidt quasi-normalised with
\beq
\int_0^{2 \pi} \int_0^{\pi} Y_{\alpha} Y_{\alpha_1} \sin \theta
d\theta d\phi= \left\{
\begin{array}{ll}
\fr{ 4 \pi }{ 2 l_{\alpha} + 1 } & \alpha = \alpha_1 \\
0  & \alpha \neq \alpha_{1}  \\
\end{array}
\right\}.
\eeq
The associated Legendre functions, $P^m_l( \cos \theta )$,
satisfy
\beq
\int_{0}^{\pi}
\left[ P_l^m ( \cos \theta )
\right]^2
\sin \theta d\theta = \fr{ 2( 2 - \delta_{m0}) }{ 2l + 1 }.
\eeq
The Laplacian operator, $\Lap$, which arises in the heat, momentum
and induction equations, is written for a scalar, $\psi$,
in spherical coordinates
\beq
\Lap \psi = \fr{1}{r^2} \pdiff{}{r}
\left( r^2 \pdiff{ \psi }{ r } \right)
+ \fr{1}{r^2 \sin \theta} \pdiff{}{\theta}
\left( \sin \theta \pdiff{ \psi }{ \theta } \right)
+ \fr{1}{ r^2 \sin^2 \theta } \pdifftw{ \psi }{ \phi }
\eeq
and by virtue of Equation (\ref{eq:alpeq}),
\beq
\Lap Y_{\alpha} = - \fr{l_{\alpha}(l_{\alpha}+1)}{r^2} Y_{\alpha}.
\eeq
If the scalar function, $\Theta(r, \theta, \phi)$, is
expanded in a series of spherical harmonics:
\beq
\Theta(r, \theta, \phi) = \sum_{\alpha} \Theta_{\alpha}(r)
Y_{\alpha},
\eeq
then
\beq
\Lap \Theta = \begin{array}{l}
\sum_{\alpha} \left[
\fr{1}{r^2} \pdiff{}{r}
\left( r^2 \pdiff{ \Theta_{\alpha}(r) }{ r } \right)
- \fr{l_{\alpha}(l_{\alpha}+1)}{r^2} \Theta_{\alpha}(r) \right]
Y_{\alpha} \\
\sum_{\alpha} {\mathcal {D} }_{l_{\alpha}} \Theta_{\alpha}(r) Y_{\alpha}
\end{array}
\eeq
The ordinary differential operator,
${\mathcal {D} }_l$, can be written in the
following equivalent forms,
\begin{eqnarray}
{\mathcal {D} }_l f &=& \fr{1}{r} \fr{d^2}{dr^2}
\left( r f \right) - \fr{ l(l+1) }{ r^2 } f 
\label{eq:dldefone} \\
   &=& \fr{1}{r^2} \fr{d}{dr} 
\left( r^2 \fr{ d f }{ dr } \right) - \fr{ l(l+1) }{ r^2 } f 
\label{eq:dldeftwo} \\
   &=& \fr{d^2 f }{dr^2 } + \fr{2}{r} \fr{d f}{dr}
- \fr{ l(l+1) }{ r^2 } f,
\label{eq:dldefthree}
\end{eqnarray}
and the composite operator ${\mathcal {D} }_l^2$, which
is found in the viscosity term of the vorticity
equation, is given by
\beq
{\mathcal {D} }_l^2 f = f^{\prime \prime \prime \prime }
+ 4 \fr{ f^{\prime \prime \prime} }{ r }
- 2l(l+1) \fr{ f^{\prime \prime } }{ r^2 }
+ (l+2)(l+1)l(l-1) \fr{ f }{ r^4 },
\label{eq:dltwodef}
\eeq
where $^{\prime}$ denotes differentiation with respect to $r$.

\section{ Poloidal and Toroidal decomposition of a solenoidal vector }

In the formalism of \cite{bullgell54}, a vector ${\bm v}$
satisfying
\beq
\nabla . {\bm v} = 0,
\eeq
can be decomposed into a toroidal and a poloidal component
\beq
{\bm v } = \sum_{\alpha} \left[
{\bm T}_{\alpha} + {\bm P}_{\alpha} \right]
\label{eq:ptseries}
\eeq
with
\beq
\begin{array}{rcll}
{\bm T}_{\alpha} &=&       &  \curl \left[ \tau_{\alpha} (r)
           P^{m_{\alpha}}_{l_{\alpha}}( \cos \theta )
           \CS_{\alpha} m \phi ~~ {\bm r } \right] \\
{\bm P}_{\alpha} &=& \curl &  \curl \left[ p_{\alpha} (r)
           P^{m_{\alpha}}_{l_{\alpha}}( \cos \theta )
           \CS_{\alpha} m \phi ~~ {\bm r } \right].
\end{array}
\label{eq:genpoltorvecs}
\eeq
%Such a representation will be referred to here as
%a 'PT-decomposition'.
As both the magnetic field and velocity (in the
Boussinesq approximation) are solenoidal vectors,
the treatment given here is general to both.
It can be easily verified that the vector
components of ${\bm T}_{\alpha}$ and 
${\bm P}_{\alpha}$ in Equation (\ref{eq:genpoltorvecs})
are given by
\beq
\begin{array}{rcl}
\left[ {\bm T}_{\alpha} \right]_r & = & 0 \\
\left[ {\bm T}_{\alpha} \right]_{\theta} & = & 
\fr{ \tau_{\alpha} (r) }{ \sin \theta }
\pdiff{ Y_{\alpha} }{ \phi }  \\
\left[ {\bm T}_{\alpha} \right]_{\phi} & = & -
 \tau_{\alpha} (r) \pdiff{ Y_{\alpha} }{ \theta } \\
& {\rm and } & \\
\left[ {\bm P}_{\alpha} \right]_r & = & 
l_{\alpha} ( l_{\alpha} + 1 ) \fr{ p_{\alpha} (r) }{ r }
Y_{\alpha} \\
\left[ {\bm P}_{\alpha} \right]_{\theta} & = & 
\left[ \fr{ p_{\alpha} (r) }{ r } + \fr{ d p_{\alpha} (r) }{d r } 
\right] \pdiff{ Y_{\alpha} }{ \theta } \\
\left[ {\bm P}_{\alpha} \right]_{\phi} & = &  
\fr{ 1 }{ \sin \theta }
\left[ \fr{ p_{\alpha} (r) }{ r } + \fr{ d p_{\alpha} (r) }{d r } 
\right] \pdiff{ Y_{\alpha} }{ \phi }.
\end{array}
\label{eq:ptcomps}
\eeq
A general vector (not necessarily solenoidal) can be represented
in a decomposition of scaloidal, spheroidal and toroidal
vector harmonics (see \cite{53morfes})
\beq
{\bm v } = \sum_{\alpha} \left[
q_{\alpha}( r ) {\bm q }_{\alpha} +
s_{\alpha}( r ) {\bm s }_{\alpha} +
t_{\alpha}( r ) {\bm t }_{\alpha}
\right]
\label{eq:qstseries}
\eeq
with
\beq
\begin{array}{ccccc}
{\bm q }_{\alpha} &=& Y_{\alpha} {\hat{\bm r}} &=&
\left[ ~~Y_{\alpha}~~,~~ 0~~,~~ 0~~ \right] \\
{\bm s }_{\alpha} &=& \fr{ 1 }{ \sqrt{ 
l_{\alpha} ( l_{\alpha} + 1 ) } }
\nabla_h ( r Y_{\alpha} ) &=&
\fr{ 1 }{ \sqrt{ 
l_{\alpha} ( l_{\alpha} + 1 ) } } \left[
0, \pdiff{ Y_{\alpha} }{ \theta } ,
\fr{ 1 }{ \sin \theta } \pdiff{ Y_{\alpha} }{ \phi }
\right] \\
{\bm t }_{\alpha} &=& \fr{ 1 }{ \sqrt{ 
l_{\alpha} ( l_{\alpha} + 1 ) } } {\bm r } \times
\nabla_h ( Y_{\alpha} ) &=&
\fr{ 1 }{ \sqrt{ 
l_{\alpha} ( l_{\alpha} + 1 ) } } \left[
0, - \fr{ 1 }{ \sin \theta } \pdiff{ Y_{\alpha} }{ \phi } ,
\pdiff{ Y_{\alpha} }{ \theta }
\right].
\end{array}
\label{eq:qstcomps}
\eeq
This representation will be referred to here as a
'qst-decomposition' and is used for calculating
the ${\bm k} \times {\bm v }$ term and non-linear
terms in the momentum and induction equations. 
Even though ${\bm v }$ and ${\bm B }$ are solenoidal,
the cross products are generally not, although because
$\curl$ is applied, the final result can be treated
as a PT-decomposed vector.

If ${\bm v}$ is a solenoidal vector, conditions on
the radial functions $q_{\alpha}(r)$, $s_{\alpha}(r)$ and
$t_{\alpha}(r)$ can be obtained to make the expansions in
(\ref{eq:ptseries}) and (\ref{eq:qstseries}) equivalent.
A direct comparison of Equations (\ref{eq:ptcomps}) and
(\ref{eq:qstcomps}) reveals
\beqar
 q_{\alpha}( r ) &=& l_{\alpha} ( l_{\alpha} + 1 ) 
\fr{ p_{\alpha}( r ) }{ r } 
\label{eq:qequiv}  \\
% \nabla . {\bm v } = 0 & \Longleftrightarrow & 
s_{\alpha}( r ) &=& \sqrt{ l_{\alpha} ( l_{\alpha} + 1 ) }
\fr{ 1 }{ r } \fr{d}{dr} \left[ r p_{\alpha}( r ) \right] 
\label{eq:sequiv}  \\
 t_{\alpha}( r ) &=&  - \sqrt{ l_{\alpha} ( l_{\alpha} + 1 ) }
\tau_{\alpha}( r ) 
\label{eq:tequiv} 
\eeqar
when $\nabla . {\bm v } = 0$.

\section{ Curls of Vectors }

Firstly we shall deal with vectors of the scaloidal,
spheroidal and toroidal type.

\subsection{ Curl of scaloidal vectors }
Let ${\bm v} = q_{\alpha} ( r ) {\bm q }_{\alpha}$.
We use the vector identity
\beq
\curl ( \psi {\bm a} ) = \psi \curl {\bm a}
- {\bm a } \times ( \nabla \psi ),
\eeq
and set the general scalar $\psi$ to be $q_{\alpha} ( r ) Y_{\alpha} / r$
and the general vector ${\bm a }$ to be the radial vector ${\bm r }$.
Hence
\beq
\curl ( q_{\alpha} ( r ) {\bm q }_{\alpha} ) = 
- {\bm r } \times \nabla_h \left( \fr{ q_{\alpha} ( r ) Y_{\alpha} }{ r }
\right)
\eeq
and comparing with the definition of the toroidal vector
in Equation (\ref{eq:qstcomps}) shows
\beq
\curl ( q_{\alpha} ( r ) {\bm q }_{\alpha} ) = 
- \sqrt{ l_{\alpha} ( l_{\alpha} + 1 ) }
\fr{ q_{\alpha} ( r )}{ r } {\bm t }_{\alpha}.
\label{eq:scalcurl}
\eeq

\subsection{ Curl of spheroidal vectors }
Let ${\bm v} = s_{\alpha} ( r ) {\bm s }_{\alpha}$.
Now
\beq
\curl ( s_{\alpha} ( r ) {\bm s }_{\alpha} ) = 
\fr{1}{\sqrt{ l_{\alpha} ( l_{\alpha} + 1 ) } }
\curl \left[ s_{\alpha} ( r ) \nabla_h ( r Y_{\alpha} ) \right]
\eeq
and
\beq
s_{\alpha} ( r ) \nabla_h ( r Y_{\alpha} ) =
\nabla \left[ r Y_{\alpha} s_{\alpha} ( r ) \right]
- {\hat {\bm r } } \left[ r s_{\alpha} ( r ) \right]^{ \prime }
Y_{\alpha}.
\eeq
Since $\curl \nabla = 0$,
\beq
\curl ( s_{\alpha} ( r ) {\bm s }_{\alpha} ) = 
- \fr{1}{\sqrt{ l_{\alpha} ( l_{\alpha} + 1 ) } }
\curl \left[ 
\fr{ \left( r s_{\alpha} \right)^{ \prime } }{ r } Y_{\alpha}
{\bm r } \right]
\eeq
By the definition (\ref{eq:genpoltorvecs}), this is
clearly a toroidal vector with 
\bedisp
\tau_{\alpha}( r ) = -\fr{1}{\sqrt{ l_{\alpha} ( l_{\alpha} + 1 ) } }
\fr{1}{r} \fr{d}{dr} \left[ r s_{\alpha} ( r ) \right]
\eedisp
and applying equivalence relation (\ref{eq:tequiv}) gives
\beq
\curl ( s_{\alpha} ( r ) {\bm s }_{\alpha} ) =
\fr{1}{r} \fr{d}{dr} \left[ r s_{\alpha} ( r ) \right]
{\bm t }_{\alpha}.
\label{eq:sphercurl}
\eeq

\subsection{ Curl of toroidal vectors }

Let ${\bm v} = t_{\alpha}( r ) {\bm t }_{\alpha}$.
It follows from Equation (\ref{eq:tequiv}) that
\beq
\curl \left[ t_{\alpha}( r ) {\bm t}_{\alpha} \right]
= \curl \curl \left[ - \fr{t_{\alpha}( r ) }
{\sqrt{ l_{\alpha} ( l_{\alpha} + 1 ) } }
Y_{\alpha} {\bm r } \right]
\eeq
which is a poloidal vector spherical harmonic with
\bedisp
p_{\alpha}( r ) = - \fr{t_{\alpha}( r ) }
{\sqrt{ l_{\alpha} ( l_{\alpha} + 1 ) } }.
\eedisp
Using the results (\ref{eq:qequiv}) and (\ref{eq:sequiv})
gives the scaloidal and spheroidal presentation of this
vector:
\beq
\curl \left[ t_{\alpha}( r ) {\bm t }_{\alpha} \right]
= - {\sqrt{ l_{\alpha} ( l_{\alpha} + 1 ) } }
\fr{ t_{\alpha} ( r ) }{ r } {\bm q }_{\alpha}
- \fr{1}{r} \fr{d}{dr} \left[ r t_{\alpha} ( r ) \right]
{\bm s }_{\alpha}
\label{eq:torcurl}
\eeq

\subsection{ Curl of poloidal vectors }

Let ${\bm v} = \curl \curl \left[ p_{\alpha} ( r ) 
Y_{\alpha} {\bm r } \right] $.

We proceed by expressing ${\bm v}$ as a
scaloidal and a spheroidal harmonic, and
applying results (\ref{eq:scalcurl}) and
(\ref{eq:sphercurl}).
\beq
{\bm v} = l_{\alpha} ( l_{\alpha} + 1 )
\fr{ p_{\alpha}( r ) }{ r } {\bm q }_{\alpha}
+ \sqrt{ l_{\alpha} ( l_{\alpha} + 1 ) }
\fr{1}{r} \fr{d}{dr} \left[ r p_{\alpha} ( r ) \right]
{\bm s }_{\alpha}
\eeq
and
\beq
\curl {\bm v} = \sqrt{ l_{\alpha} ( l_{\alpha} + 1 ) }
\left[ \fr{1}{r} \fr{d^2}{dr^2} \left( r p_{\alpha} ( r ) \right) -
l_{\alpha} ( l_{\alpha} + 1 ) \fr{ p_{\alpha} ( r ) }{ r^2 }
\right] {\bm t }_{\alpha}.
\eeq
Using (\ref{eq:dldefone}) and (\ref{eq:tequiv}) shows
that this purely toroidal vector can be written
\beq
\curl \left[ \curl \curl \left( p_{\alpha} ( r )
Y_{\alpha} {\bm r } \right) \right] =
\curl \left[
- {\mathcal {D} }_{l_{\alpha}} p_{\alpha} ( r )
Y_{\alpha} {\bm r }
\right]
\label{eq:pollapeq}
\eeq

\section{ The Laplacian }

The Laplacian of a scalar function has already been
dealt with in the introduction of the 
${\mathcal {D} }_l$ operator. For a general
vector ${\bm v} = ( v_r , v_{\theta} , v_{\phi} )$,
the Laplacian in spherical polar coordinates
(see for example \cite{95arfkenweber}) is given by
\beqar
%\begin{array}{rcl}
\left[ \Lap {\bm v } \right]_r & = &  
\Lap v_r - \fr{2}{r^2} v_r - \fr{2}{r^2} \pdiff{ v_{\theta} }{ \theta}
- \fr{ 2 \cos \theta}{ r^2 \sin \theta } v_{\theta}
- \fr{ 2 }{ r^2 \sin \theta } 
\pdiff{ v_{\phi} }{ \phi }
\label{eq:genlaprad}  \\
\left[ \Lap {\bm v } \right]_{\theta} & = & 
\Lap v_{\theta} - \fr{ 1 }{ r^2 \sin^2 \theta } v_{\theta}
+ \fr{ 2 }{ r^2 } \pdiff{ v_r }{ \theta }
- \fr{ 2 \cos \theta }{ r^2 \sin^2 \theta }
\pdiff{ v_{\phi} }{ \phi }
\label{eq:genlapthe}
   \\
\left[ \Lap {\bm v } \right]_{\phi} & = &    
\Lap v_{\phi} - \fr{ 1 }{ r^2 \sin^2 \theta } v_{ \phi }
+ \fr{2}{r^2 \sin \theta } \pdiff{ v_r }{ \phi }
+ \fr{ 2 \cos \theta }{ r^2 \sin^2 \theta }
\pdiff{ v_{\theta } }{ \phi }
%\end{array}
\label{eq:genlapphi}
\eeqar
This general formula will be used in many of the
derivations in this section.
Firstly we shall
deal with the Laplacian of vectors in the
qst-decomposition.

\subsection{ Laplacian of scaloidal vectors }

Let ${\bm v} = q_{\alpha}( r ) {\bm q }_{\alpha}$.
We use the preliminary result
\beq
\Lap \left[ \psi {\bm a } \right] =
(\Lap \psi) {\bm a } +
2 ( \nabla \psi  . \nabla ) {\bm a } +
\psi \Lap {\bm a }
\label{eq:lapprodrule}
\eeq
with the general scalar $\psi$ and general vector ${\bm a }$
replaced with $q_{\alpha}( r )$ and ${\bm q }_{\alpha}$ 
respectively. Since $q_{\alpha}( r )$ depends only upon $r$
and $Y_{\alpha}$ only on $\theta$ and $\phi$, the
cross terms are zero and so
\beq
\Lap \left[ q_{\alpha}( r ) {\bm q }_{\alpha} \right] =
\Lap \left[ q_{\alpha}( r ) \right] {\bm q }_{\alpha}
+ q_{\alpha}( r ) \Lap \left[ {\bm q }_{\alpha} \right].
\label{eq:fred}
\eeq
Being dependent only on $r$,
\beq
\Lap \left[ q_{\alpha}( r ) \right] =
\fr{ d^2 q_{\alpha}( r )}{ d r^2 } + \fr{2}{r} \fr{ d q_{\alpha}( r )}{ d r}
\label{eq:barney}
\eeq
and from (\ref{eq:genlaprad} - \ref{eq:genlapphi}),
\beq
\begin{array}{rcl}
\Lap \left[ {\bm q }_{\alpha} \right] &=&
\left[ \Lap - \fr{2}{r^2} \right] Y_{\alpha} \hat{\bm r}
+ \fr{2}{r} \nabla_h Y_{\alpha}.
\end{array}
\label{eq:wilma}
\eeq
Substituting (\ref{eq:barney}) and (\ref{eq:wilma}) into
(\ref{eq:fred}), and applying (\ref{eq:dldefthree}) and
(\ref{eq:qstcomps}), gives the relation
\beq
\Lap \left[ q_{\alpha}( r ) {\bm q }_{\alpha} \right] =
\left( {\mathcal {D}}_{l_{\alpha}} - \fr{2}{r^2} \right)
q_{\alpha}( r ) {\bm q }_{\alpha} +
2 \sqrt{ l_{\alpha} ( l_{\alpha} + 1 ) }
\fr{ q_{\alpha}( r ) }{ r^2 } {\bm s }_{\alpha}
\eeq

\subsection{ Laplacian of spheroidal vectors }
Let ${\bm v } = s_{\alpha}( r ) {\bm s }_{\alpha}$.
Using the result (\ref{eq:lapprodrule}), the
expression
\beq
\Lap \left[ s_{\alpha}( r ) {\bm s }_{\alpha} \right] =
\Lap \left[ s_{\alpha}( r ) \right] {\bm s }_{\alpha}
+ s_{\alpha}( r ) \Lap \left[ {\bm s }_{\alpha} \right].
\label{eq:pooh}
\eeq
is obtained. The Laplacian of the radial function has
exactly the same form as in (\ref{eq:barney}). For
the $\Lap \left[ {\bm s }_{\alpha} \right]$ term, 
we apply the general vector identity
\bedisp
\Lap {\bm V} = -\curl \curl {\bm V} + \nabla ( \nabla . {\bm V} ).
\eedisp
to the vector ${\bm s}_{\alpha}$:
\beq
\Lap {\bm s}_{\alpha} =
 -\curl \curl {\bm s}_{\alpha} 
+ \nabla ( \nabla . {\bm s}_{\alpha} ).
\label{eq:laststage}
\eeq
From the definition (\ref{eq:qstcomps}), 
\beq
\nabla . {\bm s}_{\alpha} = - \sqrt{ l_{\alpha} ( l_{\alpha} + 1 ) }
\fr{ Y_{\alpha} }{ r }
\label{eq:insone}
\eeq
and with
\beq
\begin{array}{rcl}
\nabla ( r Y_{\alpha} ) &=&
\left( Y_{\alpha}, \pdiff{ Y_{\alpha} }{ \theta },
\fr{ 1 }{ \sin \theta }
\pdiff{ Y_{\alpha} }{ \phi } \right) \\
&=& {\bm q }_{\alpha} +
\sqrt{ l_{\alpha} ( l_{\alpha} + 1 ) }
{\bm s }_{\alpha},
\end{array}
\eeq
we obtain
\beq
\curl \curl {\bm s }_{\alpha} =
- \fr{ \curl \curl {\bm q }_{\alpha} }{ 
\sqrt{ l_{\alpha} ( l_{\alpha} + 1 ) }
} .
\label{eq:instwo}
\eeq
Substituting (\ref{eq:insone}) and (\ref{eq:instwo}) back into
Equation (\ref{eq:laststage}) gives
\beq
\Lap {\bm s}_{\alpha} =
 \fr{ \curl \curl {\bm q}_{\alpha} }{ 
\sqrt{ l_{\alpha} ( l_{\alpha} + 1 ) } }
- \sqrt{ l_{\alpha} ( l_{\alpha} + 1 ) }
\nabla ( \fr{ Y_{\alpha} }{ r } ).
\label{eq:newstagelap}
\eeq
and therefore
\beq
\Lap {\bm s}_{\alpha} =
 \fr{ 2 \sqrt{ l_{\alpha} ( l_{\alpha} + 1 ) } }{ r^2 }
{\bm q}_{\alpha} -
\fr{ l_{\alpha} ( l_{\alpha} + 1 ) }{ r^2 }
{\bm s}_{\alpha}.
\eeq
Replacing this result back into (\ref{eq:pooh}) gives
\beq
\Lap \left[ s_{\alpha}( r ) {\bm s }_{\alpha} \right] =
{\mathcal {D} }_{l_{\alpha}} s_{\alpha}( r )
{\bm s}_{\alpha} +
2 \sqrt{ l_{\alpha} ( l_{\alpha} + 1 ) } 
\fr{ s_{\alpha}( r ) }{ r^2 }
{\bm q}_{\alpha}.
\eeq

\subsection{ Laplacian of toroidal vectors }

Let ${\bm v} = t_{\alpha} ( r ) {\bm t }_{\alpha}$.
Then
\beq
{\bm v} = \curl \left[
- \fr{ t_{\alpha}( r ) }{
\sqrt{ l_{\alpha} ( l_{\alpha} + 1 ) } }
Y_{\alpha}
{\bm r } \right],
\eeq
and since the toroidal vector is solenoidal,
\beq
\Lap {\bm v} = - \curl \curl {\bm v}
\eeq
and we proceed by taking the curl. From Equation
(\ref{eq:genpoltorvecs}) it is clear
that 
\bedisp
\curl {\bm v} = \curl \curl \left[
- \fr{ t_{\alpha}( r ) }{
\sqrt{ l_{\alpha} ( l_{\alpha} + 1 ) } }
Y_{\alpha}
{\bm r } \right],
\eedisp
and so from (\ref{eq:pollapeq}), 
\beq
\curl \curl {\bm v} = \curl \left[
- \fr{ {\mathcal {D}}_{l_{\alpha}} t_{\alpha}( r ) }{
\sqrt{ l_{\alpha} ( l_{\alpha} + 1 ) } }
Y_{\alpha}
{\bm r } \right].
\eeq
Using (\ref{eq:tequiv}),
we then have the expression for the Laplacian of both
types of toroidal vectors:
\beqar
\Lap t_{\alpha} ( r ) {\bm t }_{\alpha} &=& 
{\mathcal {D}}_{l_{\alpha}} t_{\alpha} ( r ) {\bm t }_{\alpha} \\
\Lap \left[ \curl ( \tau_{\alpha}(r) Y_{\alpha} {\bm r } ) \right]
&=& \curl ( {\mathcal {D}}_{l_{\alpha}} \tau_{\alpha}(r) Y_{\alpha} {\bm r } )
\label{eq:laptorvec}
\eeqar
The Laplacian of a toroidal vector then only involves
taking derivatives of the radial function for the same vector harmonic.

\subsection{ Laplacian of poloidal vectors }
Let ${\bm v} = \curl \curl \left[
p_{\alpha} ( r ) Y_{\alpha} {\bm r } \right]$.
Again, since the poloidal vector is solenoidal,
the $\Lap$ operator may be replaced with 
$ - \curl \curl $. Applying the $\curl$ operator twice to
${\bm v}$ and using Equation (\ref{eq:pollapeq}) gives
the result
\beq
\Lap \left[ \curl 
\curl ( p_{\alpha}(r) Y_{\alpha} {\bm r } ) \right]
= \curl 
\curl ( {\mathcal {D}}_{l_{\alpha}} p_{\alpha}(r) Y_{\alpha} {\bm r } ).
\label{eq:lappolvec}
\eeq


